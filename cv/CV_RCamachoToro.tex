% resume.tex
%
% (c) 2002 Matthew Boedicker <mboedick@mboedick.org> (original author) http://mboedick.org
% (c) 2003 David J. Grant <dgrant@ieee.org> http://www.davidgrant.ca
% (c) 2007 Todd C. Miller <Todd.Miller@courtesan.com> http://www.courtesan.com/todd
% (c) 2009-2012 Derek R. Hildreth <derek@derekhildreth.com> http://www.derekhildreth.com 
%This work is licensed under the Creative Commons Attribution-NonCommercial-ShareAlike License. To view a copy of this license, visit http://creativecommons.org/licenses/by-nc-sa/1.0/ or send a letter to Creative Commons, 559 Nathan Abbott Way, Stanford, California 94305, USA.

% GENERAL NOTE:  There may be some notes specific to myself.  If you're only interested in my LaTeX source or it doesn't make sense, please disregard it.

\documentclass[letterpaper,10pt]{article}

%-----------------------------------------------------------
\usepackage{latexsym}
\usepackage[document]{ragged2e}
\usepackage[empty]{fullpage}
\usepackage[usenames,dvipsnames]{color}
\usepackage{verbatim}
\usepackage[pdftex]{hyperref}
\hypersetup{
    colorlinks,%
    citecolor=black,%
    filecolor=black,%
    linkcolor=black,%
    urlcolor=black 
    %urlcolor=mygreylink     % can put red here to better visualize the links
}
\urlstyle{same}
\definecolor{mygrey}{gray}{.85}
\definecolor{mygreylink}{gray}{.40}
\textheight=9.0in
\raggedbottom
\raggedright
\setlength{\tabcolsep}{0in}

% Adjust margins
\addtolength{\oddsidemargin}{-0.375in}
\addtolength{\evensidemargin}{0.375in}
\addtolength{\textwidth}{0.5in}
\addtolength{\topmargin}{-.375in}
\addtolength{\textheight}{0.75in}

%-----------------------------------------------------------
%Custom commands
\newcommand{\resitem}[1]{\item #1 \vspace{-2pt}}
\newcommand{\resheading}[1]{{\large \colorbox{mygrey}{\begin{minipage}{\textwidth}{\textbf{#1 \vphantom{p\^{E}}}}\end{minipage}}}}
\newcommand{\ressubheading}[4]{
\begin{tabular*}{6.5in}{l@{\extracolsep{\fill}}r}
		\textbf{#1} & #2 \\
		\textit{#3} & \textit{#4} \\
\end{tabular*}\vspace{-6pt}}

\newcommand{\ressubsubheading}[2]{
\begin{tabular*}{6.5in}{l@{\extracolsep{\fill}}r}
		\textit{#1} & \textit{#2} \\
\end{tabular*}\vspace{-6pt}}
%-----------------------------------------------------------

%-----------------------------------------------------------
%General Resume Tips
%   No periods!  Technically, nothing in this document is a full sentence.
%   Use parallelism by ending key words with the same thing,  i.e. "Coordinated; Designed; Communicated".
%   More tips on bottom of this LaTeX document.
%-----------------------------------------------------------

\begin{document}



\newcommand{\mywebheader}{
\begin{tabular*}{7in}{l@{\extracolsep{\fill}}r}
	\textbf{}{\LARGE Reina CAMACHO TORO} \\
	{\footnotesize \texttt{  Date of birth: 07/09/1987}} & Email: \href{mailto:reina.camacho@cern.ch}{reina.camacho@cern.ch}\\
	{\footnotesize \texttt{  Nationality: Venezuelan}}  & Date: March 2019\\
	\end{tabular*}
\\
\vspace{0.1in}}



% CHANGE HEADER SOURCE HERE
\mywebheader


%%%%%%%%%%%%%%%%%%%%%%
\resheading{Professional profile}


\justify{\textbf{Particle physicist}, understanding models and theories about how the universe work. Comprehensive background in data analysis, mathematical modelling and statistical analysis. R\&D of silicon particle detectors and online event selection systems. Extensive experience in \textbf{science communication and promotion of education initiatives} for social progress. Exceptional leadership in managing complex scientific projects in \textbf{multi-disciplinary and international collaborative teams}, such as the \href{http://atlas.ch/}{\color{blue}{ATLAS experiment}} of the Large Hadron Collider (\href{http://home.cern/topics/large-hadron-collider}{\color{blue}{LHC}}) at \href{http://home.cern/}{\color{blue}{CERN}}, the European Organisation for Nuclear Research.  Excellent communication skills developed through technical discussions and presentations to audiences ranging from field experts to the general public.}\\

\noindent
%%%%%%%%%%%%%%%%%%%%%%
\resheading{Employment/ Professional experience}
	\begin{itemize}
	        \item 
			\ressubheading{\href{https://lpnhe.in2p3.fr}{French National Center for Scientific Research (CNRS), LPNHE}}{Paris, France}
				{Permanent researcher, ATLAS experiment at CERN}{Jan. 2018 -- Present}
		\item 
			\ressubheading{\href{http://hep.uchicago.edu/atlas/}{University of Chicago, Enrico Fermi Institute}}{Chicago, US}
				{Post-doctoral researcher, ATLAS experiment at CERN}{Feb. 2015 -- Jan. 2018}

		\item 
			\ressubheading{\href{http://dpnc.unige.ch}{Universit\'e de Gen\`eve}}{Geneva, Switzerland}
			{Post-doctoral researcher, ATLAS experiment at CERN}{Jan. 2013 -- Jan. 2015}
			
		\item 
			\ressubheading{\href{https://home.cern/students-educators/summer-student-programme}{CERN}}{Geneva, Switzerland}
			{Summer student program at CERN}{Jun. 2009 -- Sep. 2009}		
	\end{itemize}  % End Experience list

\noindent
\resheading{Education}
	\begin{itemize}
		\item
			\ressubheading{Universit{\'e} Blaise Pascal, LPC}{Clermont Ferrand, France}{PhD in Particle Physics}{Nov. 2009 --  Nov. 2012}
		\item 	
			\ressubheading{Universidad de Los Andes}{M\'erida, Venezuela}{Bachelor in Physics}{Sept. 2003 --  Apr. 2008}\\

	\end{itemize} % End Education list

\noindent
\resheading{Positions, responsabilites and honors received}
	\begin{itemize}
		\item 2019-Present, Member of the Preparatory Group for the \textbf{Latin American Strategy Forum for Research Infrastructure (LASF4RI)}
		\item 2019-Present, Member of the \textbf{ATLAS Early Career Scientists Board (ECSB)}. A board to support early-career physicists on ATLAS, and to prepare them for future roles in academia or beyond
		\item 2018-Present, Member of the \textbf{Organization for Women in Science for the Developing World (OWSD)}
		\item 2018-Present, Member of the \textbf{Venezuelan Physics Society}
	        \item 2016-Present, \textbf{Coordinator} of the \href{https://www.ictp.it/physics-without-frontiers/current-country-projects/venezuela-and-colombia.aspx}{\textcolor{blue}{Physics without Frontiers program for Latin-America}}. This is a program from the International Centre for Theoretical Physics (ICTP) that brings physics to under-represented people around the world. The team has organised projects in many countries around the world. So far I have planned and led activities in Venezuela, Colombia, Peru, Ecuador, Uruguay and Argentina
		\item 2014-Present, \textbf{Co-founder} ( and director between 2014 and 2016) of the \href{http://www.cevale2ve.org}{\textcolor{blue}{CEVALE2VE}} (Centro Virtual de Altos Estudios de Altas Energias) group. CEVALE2VE promotes of high energy physics in Latin-America through webinars, virtual courses and career \& opportunities sessions. Partnerships with institutions in Venezuela, Colombia, Ecuador and Peru.
		\item 2017, Invited to participate in the \textbf{International Visitor Leadership Program (iVLP) of U.S. Department of State: Translating Science, Fostering Innovation}
		\item 2017, Participant in the \textbf{TWAS-AAAS Science Diplomacy Course} in Trieste, Italy
		\item 2016-2017, \textbf{Coordinator of the Diboson-Exotics subgroup within the ATLAS Collaboration}, formed by \textbf{100 persons} analyzing huge datasets devoted to new physics searches
		\item 2015-2016, \textbf{Coordinator of the boosted boson tagging performance group within the ATLAS Collaboration}, formed by \textbf{30 persons} from different institutions and universities around the world working on understanding how to identify boosted bosons with the ATLAS detector
		\item 2008-2009, \textbf{High Energy Physics Latinamerican-European Network (HELEN) fellowship}, LPNHE Université Pierre et Marie Curie-Paris VI, Paris, France
		
	\end{itemize} 

\noindent
\resheading{Teaching experience and student supervision (PhD and undergraduate)}
	\begin{itemize}
		\item 2014-Present, Online introductory course of particle physics from CEVALE2VE organization. In collaboration with 4 Venezuelan, 2 Colombian and one Peruvian institutions
		\item 2013-2015, Postdoctoral teaching assistant at Universit{\'e} de Gen{\`e}ve, Geneva, Switzerland
		\item 2005-2008, Undergraduate teaching assistant at Universidad de Los Andes, M{\'e}rida, Venezuela
		\item Supervision of 4 undergraduate students, 3 master and 3 PhD students during my postdocs and/or my current researcher position
	\end{itemize} 

\noindent
\resheading{Main selected publications}

In this section I want to give an overview of my scientific activities to allow the reviewers to have a better insight into the list of publication below. I am an experimental particle physicist working at the \href{http://atlas.ch/}{\color{blue}{ATLAS experiment}} of the Large Hadron Collider (\href{http://home.cern/topics/large-hadron-collider}{\color{blue}{LHC}}) at \href{http://home.cern/}{\color{blue}{CERN}}. The ATLAS Collaboration consists of around 3000 researchers and engineers from 38 countries and this is why our papers and conferences notes are signed by all the collaborators, you will see ``ATLAS Collaboration'' very often in the author field. I have more than 200 papers published as ATLAS author but below you will see a selected list of publications where I had a major/leading/coordination contribution. 

The LHC is the most powerful machine ever built on earth, a 27 km ring that accelerates protons to a speed close to the speed of light and collides them. In each of this energetic collisions (we have around one billion of collisions per second!) we recreate the conditions at the beginning of the universe, little Big Bangs. ATLAS investigates many different types of physics that might become detectable in the energetic collisions of the LHC. Some of members of the collaboration work on confirmations or improved measurements of the Standard Model theory like the recently discovered Higgs boson, also known as God particle. Others look for possible clues for new physical theories that explains all the nature unknowns like the existence of Dark Matter and Dark Energy that makes up for 95\% of the know universe. And many others work in making sure the detector works and is taking data or developing new instruments to upgrade the current system. In my case I work on analyzing data to look for new physics clues and on R\&D of silicon particle detectors (project known as Inner Tracker detector) and online event selection system (project known as global Feature EXtractor, gFEX) for the upgrade of our detector for future accelerator runs which is critical, as we will be collecting data from particle collisions at a much higher rate than ever before. The publications listed below are related to this work. 

%{ \footnotesize
	\begin{itemize}
	       \item ATLAS Collaboration. ``Combination of searches for heavy resonances decaying into bosonic and leptonic final states using 36/fb of proton-proton collision data at $\sqrt{s}=13$ TeV with the ATLAS detector''. \href{https://arxiv.org/abs/1808.02380}{\color{blue}{hep-ex/1808.02380}}. CERN-EP-2018-179. submitted to Phys. Rev. D.
	       
	       \item ATLAS Collaboration. ``Technical Design Report for the ATLAS Inner Tracker Pixel Detector''. \href{https://cds.cern.ch/record/2285585}{\color{blue}{CERN-LHCC-2017-021}}. 
	        
	        \item ATLAS Collaboration. ``gFEX, the ATLAS Calorimeter Global Feature Extractor for the Phase-I upgrade of the ATLAS experiment''. \href{https://pos.sissa.it/282/1055}{\color{blue}{DOI:10.22323/1.282.1055}}. Volume 282 - 38th International Conference on High Energy Physics (ICHEP2016).
	        
		\item ATLAS Collaboration. ``Search for diboson resonances with boson-tagged jets in
                        $pp$ collisions at $\sqrt{s}=13$ TeV with the ATLAS
                        detector''. \href{https://arxiv.org/abs/1708.04445}{\color{blue}{hep-ex/1708.04445}}. CERN-EP-2017-147. Phys. Lett. B 777 (2017) 91.
	
	        \item ATLAS Collaboration. ``Searches for heavy diboson resonances in $pp$ collisions at $\sqrt{s}=13 $~TeV with the ATLAS detector''. \href{http://arxiv.org/abs/1606.04833}{\color{blue}{hep-ex/1606.04833 }}. CERN-EP-2016-106. JHEP 09 (2016) 173.
		\item ATLAS Collaboration. ``Identification of boosted, hadronically decaying $W$ bosons and comparisons with ATLAS data taken at $\sqrt{s}=8 $~TeV''. \href{http://arxiv.org/abs/1510.05821}{\color{blue}{hep-ex/1510.05821}}. CERN-PH-EP-2015-204. 
{\bf Eur. Phys. J. C 76(3) (2016) 1-47} 
                \item ATLAS Collaboration. ``Search for high-mass diboson resonances with boson-tagged jets in proton-proton collisions at $\sqrt{s}= 8$~TeV with the ATLAS detector''. \href{http://arxiv.org/abs/1506.00962}{\color{blue}{hep-ex/1506.00962}}. CERN-PH-EP-2015-115. 
{\bf JHEP 12 (2015) 55}
	      	\item ATLAS Collaboration. ``Observation of a new particle in the search for the Standard Model Higgs boson with the ATLAS detector at the LHC". 
  \href{https://arxiv.org/abs/1207.7214}{\color{blue}{hep-ex/1207.7214}}. CERN-PH-EP-2012-218. Phys.Lett. B716 (2012) 1-29.
  
		\item ATLAS Collaboration. ``Jet energy measurement with the ATLAS detector in $pp$ collisions 
at $\sqrt{s} = 7$~TeV''. \href{http://arxiv.org/abs/1112.6426}{\color{blue}{hep-ex/1112.6426}}. CERN-PH-EP-2011-191. 
{Eur. Phys. J. C, 73 3 (2013) 2304}

	\end{itemize} % End main publications
	
	Citation rate and full list of publication as author of the ATLAS collaboration can be found with my ORCID identifier 0000-0002-9192-8028.
	
	%}

\noindent
\resheading{Selected presentations in conferences and seminars}

%{ \footnotesize
	\begin{itemize}
	        \item ``SUSY searches - electroweak production''. \href{https://susy2018.ifae.es}{\textcolor{blue}{26th International Conference on Supersymmetry and Unification of Fundamental Interactions (SUSY2018)}}. \href{http://cds.cern.ch/record/2632810/files/ATL-PHYS-SLIDE-2018-561.pdf}{\textcolor{blue}{ATL-PHYS-SLIDE-2018-561}}. July 23rd-27th, 2018. Barcelone, Spain.
	      	\item ``Outreaching particle physics to Latin America: CEVALE2VE and the use of ATLAS open data ''. \href{http://inspirehep.net/record/1664808}{\textcolor{blue}{EPS Conference on High Energy Physics}}. July 5th-12th, 2017. Venice, Italy.
		\item ``Measurements and searches with boosted techniques''. \href{https://indico.cern.ch/event/456448/contributions/2274584/attachments/1376918/2091182/PIC2016_Camacho_v2.pdf}{\textcolor{blue}{Particle In Collisions (PIC)}}. September 14th-17th, 2016. Quy Nhon, Vietnam.
		\item ``ATLAS Higgs physics prospects at the high luminosity LHC''. 
     \href{http://indico.ific.uv.es/indico/conferenceDisplay.py?ovw=True&confId=2025}{ ICHEP}, 
     July 2nd-9th, 2014. Valencia, Spain. \href{https://cds.cern.ch/record/1742973}{\color{blue}{ATL-PHYS-SLIDE-2014-424}}
     		\item ``Higgs coupling studies at a High Luminosity-LHC with ATLAS detector''. 
    \href{http://www.sps.ch/en/events/sps_annual_meeting_2014/}{\color{blue}{Swiss Physics Society (SPS) Meeting 2014}}, 30 Juin-2 July, Fribourg, Switzerland
        		%\item ``LHC and ATLAS : in the heart of the matter'', all public seminar at the Institute of Scientific Investigations of Venezuela (IVIC). December 6th, 2012.
      %San Antonio de Los Altos, Venezuela.
     		\item ``Searches for $t\bar{t}$ resonances in ATLAS''. 
     \href{http://confs.obspm.fr/RencontresVietnam/BTSM/index.htm}{Beyond The Standard Model of Particle Physics}, 
     July 15th-21th, 2012. Quy Nhon, Vietnam. \href{https://cds.cern.ch/record/1471377}{\color{blue}{ATL-PHYS-SLIDE-2012-487}}
     		\item 2012 GDR Terascale. ``Top-antitop resonances at the LHC''. 23-25 Avril, 2012. Clermont
Ferrand, France.
     		\item Invitation to Particle Physics Seminars: University of Dresdes (Germany), Technion (Israel), University of Bonn (Germany), CEA Saclay (France), University of Chicago (US),  SLAC National Accelerator Laboratory (US), Harvard University (US), Brookhaven National Laboratory-BNL (US), Argonne National Laboratory-ANL (US), Institute of Scientific Investigations of Venezuela-IVIC (Venezuela), Universidad de Los Andes-ULA (Venezuela)
	\end{itemize} % End selected public presentations
%}

\noindent
\resheading{List of organised scientific conferences and workshops}

%{ \footnotesize
	\begin{itemize}
	        \item 2018, BOOST conference, 10th International Workshop on Boosted Object Phenomenology, Reconstruction and Searches in High Energy Physics. Local organisation committee. 16-20 July, 2018, Paris, France. Currently part of the International Advisory Committee. 
		\item 2017, ATLAS Exotics and SUSY Joint Workshop. Organiser/convener of the SUSY and searches with bosons sessions. May 8- 12th, 2017. Bucharest, Romania. 
		\item 2016, ATLAS Boosted Object Tagging (ABOT). Organiser/convener of the Higgs Tagging
session. April 20- 22th, 2016. Heidelberg, Germany.
		\item 2015, ATLAS Hadronic Calibration Workshop (HCW). Organiser/convener of the Jet
substructure and Hadronic Tagging session. Septembre 14-18th, 2015. Bratislava,
Slovakia.
     		%\item 2015, BOOST conference, 7th International Workshop on Boosted Object Phenomenology, Reconstruction and Searches in High Energy Physics. Local organisation committee. 10-14 August, 2015, Chicago, US.
	\end{itemize} % End selected public presentations
%}

\noindent
\resheading{Skills}
	\begin{description}
		\item[Languages:] { \footnotesize Spanish (native), English (full professional proficiency/C1), French (full professional proficiency/B2-C1) and Italian (full professional proficiency/B2-C1)
		}
		\item[IT proficiency:] { \footnotesize C++, ROOT, Python, RooFit, TMVA-Toolkit for Multivariate Data Analysis, LaTeX, Microsoft Office, Windows and Linux operating systems, e-learning tools and social media as educational tools.
		}
		\item[Science:] { \footnotesize
			Computer science, physics, statistical analyses, data science, Monte Carlo simulation, quantitative research
		}
		\item[Soft skills:] { \footnotesize
			Analytical, efficient, accurate, independent, flexible, communicative, creative, reliable, enthusiastic, conflict resolution.
		}
	\end{description} % End Skills list


\end{document}


%%%%%%%%%%%%%%%%%%%%%%%%
% GENERAL RESUME NOTES %
%%%%%%%%%%%%%%%%%%%%%%%%
Need to brush up on the skills section?  Here are some really good ideas:

- Communication: The ability to express and interpret ideas and convey knowledge. 
Skills like speaking effectively, writing concisely, listening attentively, expressing ideas, 
reporting information, editing, interviewing, and facilitating group discussion. 
 
- Research and Planning: The ability to search for specific knowledge and formulate a 
program for a definite course of action. Skills like forecasting, predicting, identifying 
issues, finding alternatives, gathering information, solving problems, setting goals, 
extracting information, and developing strategies. 
 
- Human Relations: The ability to apply interpersonal skills to resolve conflict, relate to 
people, and help people. Skills like providing support for others, listening, delegating 
with respect, representing others, asserting, developing rapport, and perceiving 
feelings. 
 
- Management, Organization, Leadership: The ability to supervise others and guide 
individuals and groups towards the completion of tasks. Skills like managing groups, 
selling ideas, making decisions, managing conflict, coordinating tasks, teaching, 
enlisting help. 

Source: 
http://www.squawkfox.com/2009/03/08/6-action-words-that-make-your-resume-rock/

Other Resources: 
http://www.tvmier.com/Resume_Education.html

